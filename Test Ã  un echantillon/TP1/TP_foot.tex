% Options for packages loaded elsewhere
\PassOptionsToPackage{unicode}{hyperref}
\PassOptionsToPackage{hyphens}{url}
%
\documentclass[
]{article}
\usepackage{amsmath,amssymb}
\usepackage{iftex}
\ifPDFTeX
  \usepackage[T1]{fontenc}
  \usepackage[utf8]{inputenc}
  \usepackage{textcomp} % provide euro and other symbols
\else % if luatex or xetex
  \usepackage{unicode-math} % this also loads fontspec
  \defaultfontfeatures{Scale=MatchLowercase}
  \defaultfontfeatures[\rmfamily]{Ligatures=TeX,Scale=1}
\fi
\usepackage{lmodern}
\ifPDFTeX\else
  % xetex/luatex font selection
\fi
% Use upquote if available, for straight quotes in verbatim environments
\IfFileExists{upquote.sty}{\usepackage{upquote}}{}
\IfFileExists{microtype.sty}{% use microtype if available
  \usepackage[]{microtype}
  \UseMicrotypeSet[protrusion]{basicmath} % disable protrusion for tt fonts
}{}
\makeatletter
\@ifundefined{KOMAClassName}{% if non-KOMA class
  \IfFileExists{parskip.sty}{%
    \usepackage{parskip}
  }{% else
    \setlength{\parindent}{0pt}
    \setlength{\parskip}{6pt plus 2pt minus 1pt}}
}{% if KOMA class
  \KOMAoptions{parskip=half}}
\makeatother
\usepackage{xcolor}
\usepackage[margin=1in]{geometry}
\usepackage{color}
\usepackage{fancyvrb}
\newcommand{\VerbBar}{|}
\newcommand{\VERB}{\Verb[commandchars=\\\{\}]}
\DefineVerbatimEnvironment{Highlighting}{Verbatim}{commandchars=\\\{\}}
% Add ',fontsize=\small' for more characters per line
\usepackage{framed}
\definecolor{shadecolor}{RGB}{248,248,248}
\newenvironment{Shaded}{\begin{snugshade}}{\end{snugshade}}
\newcommand{\AlertTok}[1]{\textcolor[rgb]{0.94,0.16,0.16}{#1}}
\newcommand{\AnnotationTok}[1]{\textcolor[rgb]{0.56,0.35,0.01}{\textbf{\textit{#1}}}}
\newcommand{\AttributeTok}[1]{\textcolor[rgb]{0.13,0.29,0.53}{#1}}
\newcommand{\BaseNTok}[1]{\textcolor[rgb]{0.00,0.00,0.81}{#1}}
\newcommand{\BuiltInTok}[1]{#1}
\newcommand{\CharTok}[1]{\textcolor[rgb]{0.31,0.60,0.02}{#1}}
\newcommand{\CommentTok}[1]{\textcolor[rgb]{0.56,0.35,0.01}{\textit{#1}}}
\newcommand{\CommentVarTok}[1]{\textcolor[rgb]{0.56,0.35,0.01}{\textbf{\textit{#1}}}}
\newcommand{\ConstantTok}[1]{\textcolor[rgb]{0.56,0.35,0.01}{#1}}
\newcommand{\ControlFlowTok}[1]{\textcolor[rgb]{0.13,0.29,0.53}{\textbf{#1}}}
\newcommand{\DataTypeTok}[1]{\textcolor[rgb]{0.13,0.29,0.53}{#1}}
\newcommand{\DecValTok}[1]{\textcolor[rgb]{0.00,0.00,0.81}{#1}}
\newcommand{\DocumentationTok}[1]{\textcolor[rgb]{0.56,0.35,0.01}{\textbf{\textit{#1}}}}
\newcommand{\ErrorTok}[1]{\textcolor[rgb]{0.64,0.00,0.00}{\textbf{#1}}}
\newcommand{\ExtensionTok}[1]{#1}
\newcommand{\FloatTok}[1]{\textcolor[rgb]{0.00,0.00,0.81}{#1}}
\newcommand{\FunctionTok}[1]{\textcolor[rgb]{0.13,0.29,0.53}{\textbf{#1}}}
\newcommand{\ImportTok}[1]{#1}
\newcommand{\InformationTok}[1]{\textcolor[rgb]{0.56,0.35,0.01}{\textbf{\textit{#1}}}}
\newcommand{\KeywordTok}[1]{\textcolor[rgb]{0.13,0.29,0.53}{\textbf{#1}}}
\newcommand{\NormalTok}[1]{#1}
\newcommand{\OperatorTok}[1]{\textcolor[rgb]{0.81,0.36,0.00}{\textbf{#1}}}
\newcommand{\OtherTok}[1]{\textcolor[rgb]{0.56,0.35,0.01}{#1}}
\newcommand{\PreprocessorTok}[1]{\textcolor[rgb]{0.56,0.35,0.01}{\textit{#1}}}
\newcommand{\RegionMarkerTok}[1]{#1}
\newcommand{\SpecialCharTok}[1]{\textcolor[rgb]{0.81,0.36,0.00}{\textbf{#1}}}
\newcommand{\SpecialStringTok}[1]{\textcolor[rgb]{0.31,0.60,0.02}{#1}}
\newcommand{\StringTok}[1]{\textcolor[rgb]{0.31,0.60,0.02}{#1}}
\newcommand{\VariableTok}[1]{\textcolor[rgb]{0.00,0.00,0.00}{#1}}
\newcommand{\VerbatimStringTok}[1]{\textcolor[rgb]{0.31,0.60,0.02}{#1}}
\newcommand{\WarningTok}[1]{\textcolor[rgb]{0.56,0.35,0.01}{\textbf{\textit{#1}}}}
\usepackage{graphicx}
\makeatletter
\def\maxwidth{\ifdim\Gin@nat@width>\linewidth\linewidth\else\Gin@nat@width\fi}
\def\maxheight{\ifdim\Gin@nat@height>\textheight\textheight\else\Gin@nat@height\fi}
\makeatother
% Scale images if necessary, so that they will not overflow the page
% margins by default, and it is still possible to overwrite the defaults
% using explicit options in \includegraphics[width, height, ...]{}
\setkeys{Gin}{width=\maxwidth,height=\maxheight,keepaspectratio}
% Set default figure placement to htbp
\makeatletter
\def\fps@figure{htbp}
\makeatother
\setlength{\emergencystretch}{3em} % prevent overfull lines
\providecommand{\tightlist}{%
  \setlength{\itemsep}{0pt}\setlength{\parskip}{0pt}}
\setcounter{secnumdepth}{-\maxdimen} % remove section numbering
\ifLuaTeX
  \usepackage{selnolig}  % disable illegal ligatures
\fi
\IfFileExists{bookmark.sty}{\usepackage{bookmark}}{\usepackage{hyperref}}
\IfFileExists{xurl.sty}{\usepackage{xurl}}{} % add URL line breaks if available
\urlstyle{same}
\hypersetup{
  pdftitle={TP1 : 1 echantillon},
  hidelinks,
  pdfcreator={LaTeX via pandoc}}

\title{TP1 : 1 echantillon}
\author{}
\date{\vspace{-2.5em}2024-09-24}

\begin{document}
\maketitle

\hypertarget{fonctions-r-utiles-pour-ce-tp}{%
\section{Fonctions R utiles pour ce TP
:}\label{fonctions-r-utiles-pour-ce-tp}}

Dans ce TP, vous aurez besoin des fonctions suivantes pour réaliser les
tests statistiques :

\begin{itemize}
\tightlist
\item
  \textbf{\texttt{t.test()}} : Effectue un test de Student pour comparer
  une moyenne observée à une moyenne théorique.
\item
  \textbf{\texttt{power.t.test()}} : Calcule la puissance d'un test de
  Student
\item
  \textbf{\texttt{binom.test()}} : Réalise un test exact sur le
  paramètre d'une loi de Bernoulli (utile pour tester une proportion).
\item
  \textbf{\texttt{prop.test()}} : Effectue un test approximatif (via une
  approximation normale) sur le paramètre d'une loi de Bernoulli.
\end{itemize}

\hypertarget{pour-ruxe9aliser-les-tests-uxe0-la-main-vous-aurez-uxe9galement-besoin-des-fonctions-suivantes}{%
\subsubsection{Pour réaliser les tests ``à la main'', vous aurez
également besoin des fonctions suivantes
:}\label{pour-ruxe9aliser-les-tests-uxe0-la-main-vous-aurez-uxe9galement-besoin-des-fonctions-suivantes}}

\begin{itemize}
\tightlist
\item
  \textbf{\texttt{pt(x,\ n)}} : Renvoie la probabilité qu'une variable
  de Student à \(n\) degrés de liberté soit inférieure à \(t = x\).
  Utile pour calculer la p-value lors d'un test t.
\item
  \textbf{\texttt{qt(alpha,\ n)}} : Renvoie le quantile \(\alpha\) pour
  une loi de Student à \(n\) degrés de liberté. Utile pour trouver la
  valeur critique dans un test t.
\end{itemize}

\hypertarget{conseil}{%
\subsubsection{Conseil :}\label{conseil}}

N'hésitez pas à consulter l'aide de ces fonctions avec \texttt{?t.test},
\texttt{?power.t.test}, \texttt{?binom.test}, etc., pour en savoir plus
sur leur utilisation et leurs options.

\hypertarget{ligue-1}{%
\section{Ligue 1:}\label{ligue-1}}

Nous avons une base de données nommée ligue1\_2425 qui donne les
résultats des matchs de la saison 2024-2025 de la Ligue 1. À ce jour, 5
journées ont eu lieu (chaque équipe a donc joué 5 matchs).

La base de données contient 4 colonnes :

\begin{itemize}
\tightlist
\item
  Dom (l'équipe jouant à domicile),
\item
  Dom\_but (nombre de buts marqués par l'équipe à domicile),
\item
  Ext (l'équipe jouant à l'extérieur),
\item
  Ext\_but (nombre de buts marqués par l'équipe à l'extérieur).
\end{itemize}

\hypertarget{nombre-de-but-psg}{%
\subsection{Nombre de but (PSG)}\label{nombre-de-but-psg}}

L'entraîneur du PSG pense que son équipe est devenue significativement
meilleure que la saison dernière et qu'à partir du début de cette saison
(août 2024), elle marquera en moyenne plus de buts par match de Ligue 1.

L'année dernière, le PSG avait marqué en moyenne 2.38 buts par match.
Pour l'instant, nous n'avons que les résultats des 5 premières journées
de Ligue 1, ce qui correspond à 5 matchs du PSG.

Soit \(X_i\) le nombre de buts marqués par le PSG pendant le match de la
journée \(i\). Nous supposons que les \(X_i\) sont i.i.d (indépendants
et identiquement distribués), avec \(E(X_i) = \mu\) et
\(V(X_i) = \sigma^2\).

\hypertarget{mieux-que-lannuxe9e-derniuxe8re}{%
\subsubsection{Mieux que l'année dernière
?}\label{mieux-que-lannuxe9e-derniuxe8re}}

\begin{enumerate}
\def\labelenumi{\arabic{enumi}.}
\tightlist
\item
  Représentez graphiquement le nombre de buts marqués par le PSG au
  cours de ses matchs de Ligue 1.
\end{enumerate}

\begin{Shaded}
\begin{Highlighting}[]
\CommentTok{\# lire le fichier de données}
\FunctionTok{load}\NormalTok{(}\AttributeTok{file =} \StringTok{"ligue1\_2425.Rdata"}\NormalTok{) }

\CommentTok{\# Pour récupérer le nombre de but du PSG : }

\NormalTok{but\_psg }\OtherTok{\textless{}{-}}\NormalTok{ ligue1\_2425 }\SpecialCharTok{\%\textgreater{}\%} 
  \FunctionTok{filter}\NormalTok{(Ext }\SpecialCharTok{==}\StringTok{"PSG"} \SpecialCharTok{|}\NormalTok{ Dom }\SpecialCharTok{==}\StringTok{"PSG"}\NormalTok{) }\SpecialCharTok{\%\textgreater{}\%} 
  \FunctionTok{mutate}\NormalTok{(}\AttributeTok{but\_psg =} \FunctionTok{case\_when}\NormalTok{(Dom }\SpecialCharTok{==}\StringTok{"PSG"}\SpecialCharTok{\textasciitilde{}}\NormalTok{ Dom\_but,}
                             \ConstantTok{TRUE}\SpecialCharTok{\textasciitilde{}}\NormalTok{ Ext\_but)) }\SpecialCharTok{\%\textgreater{}\%} 
  \FunctionTok{pull}\NormalTok{(but\_psg) }\SpecialCharTok{\%\textgreater{}\%} \FunctionTok{as.numeric}\NormalTok{()}
\end{Highlighting}
\end{Shaded}

\begin{enumerate}
\def\labelenumi{\arabic{enumi}.}
\setcounter{enumi}{1}
\item
  Comme nous n'avons qu'un échantillon de taille \(n = 5\), quelle
  hypothèse supplémentaire devons-nous faire sur les \(X_i\) pour
  pouvoir effectuer un des tests statistiques vu en cours?
\item
  En supposant ce qu'il faut (réponse à la question 2), mettez en place
  un test au niveau de significativité de 5 \% pour tester si la moyenne
  des buts par match du PSG diffère de celle de l'année dernière (2.38
  buts). Effectuez ce test en utilisant la fonction adéquate et en le
  faisant étape par étape (en décrivant la zone de rejet, calculant la
  pvaleur etc)
\item
  Pouvons-nous conclure que l'équipe n'est pas meilleure ? Que
  faudrait-il analyser en plus avant d'arriver à cette conclusion ?
\end{enumerate}

\hypertarget{puissance-du-test}{%
\subsubsection{Puissance du test}\label{puissance-du-test}}

L'entraîneur pense que la vraie nouvelle moyenne de buts est 3.5 et non
2.38.

\begin{enumerate}
\def\labelenumi{\arabic{enumi}.}
\setcounter{enumi}{4}
\item
  S'il a raison, quelle est la puissance du test pour détecter cette
  différence ? Autrement dit, quelle est la probabilité de mettre en
  évidence, avec nos données, une différence de 1.12 entre la moyenne de
  l'année dernière et l'espérance actuelle des buts marqués par le PSG ?
\item
  Actuellement, nous n'avons que 5 matchs, mais d'autres matchs seront
  joués au fur et à mesure. Comment la puissance du test évolue-t-elle
  avec l'augmentation du nombre de matchs disponibles (augmente,
  diminue, reste inchangée) ?
\item
  Combien de matchs du PSG (après août 2024) faudrait-il pour mettre en
  évidence cette différence ( de 1.12) avec une puissance de 90 \% ?
\item
  Tracez une courbe représentant la puissance du test en fonction du
  nombre de matchs disponibles.
\item
  L'entraîneur pense finalement que la vraie nouvelle moyenne de buts
  est de 3, et non 3.5. Cela change-t-il la puissance du test ? Si oui,
  comment ? Recalculez-la.
\item
  Tracez la puissance du test en fonction de l'écart entre l'ancienne
  moyenne et la nouvelle moyenne. Par exemple, si la nouvelle moyenne
  est de 3, l'écart est de 0.62; si elle est de 3.5, l'écart est de 1.12
  etc. Calculer là pour des écarts allant de 0.2 à 4.
\end{enumerate}

\hypertarget{proba-de-gagner-psg}{%
\subsection{Proba de gagner (PSG)}\label{proba-de-gagner-psg}}

Finalement, l'entraîneur réfléchit. Ce qui l'intéresse réellement, c'est
la probabilité de gagner un match. L'année dernière, le PSG a gagné 64\%
de ses matchs. L'entraîneur souhaite montrer, avec un risque de 5 \%,
que cette proportion est significativement plus élevée depuis août 2024.
Réalisez le test.

\hypertarget{ligue-1-nombre-de-but-toulouse}{%
\subsection{Ligue 1: nombre de but
(Toulouse)}\label{ligue-1-nombre-de-but-toulouse}}

L'entraineur de Toulouse pense que son équipe marque significativement
moins de but que l'année dernière. L'année dernière l'équipe marquait
1.23 buts par match en moyenne. En supposant la normalité du nombre de
buts marqués par match, réalisez un test au niveau 5\% pour répondre à
l'entraineur.

L'entraineur pense que la nouvelle espérance de but de l'équipe de
Toulouse est de 0.9. Calculer la puissance du test pour mettre en
évidence une telle différence.

\end{document}
