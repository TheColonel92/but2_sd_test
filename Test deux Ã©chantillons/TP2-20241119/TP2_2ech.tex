% Options for packages loaded elsewhere
\PassOptionsToPackage{unicode}{hyperref}
\PassOptionsToPackage{hyphens}{url}
%
\documentclass[
]{article}
\usepackage{amsmath,amssymb}
\usepackage{iftex}
\ifPDFTeX
  \usepackage[T1]{fontenc}
  \usepackage[utf8]{inputenc}
  \usepackage{textcomp} % provide euro and other symbols
\else % if luatex or xetex
  \usepackage{unicode-math} % this also loads fontspec
  \defaultfontfeatures{Scale=MatchLowercase}
  \defaultfontfeatures[\rmfamily]{Ligatures=TeX,Scale=1}
\fi
\usepackage{lmodern}
\ifPDFTeX\else
  % xetex/luatex font selection
\fi
% Use upquote if available, for straight quotes in verbatim environments
\IfFileExists{upquote.sty}{\usepackage{upquote}}{}
\IfFileExists{microtype.sty}{% use microtype if available
  \usepackage[]{microtype}
  \UseMicrotypeSet[protrusion]{basicmath} % disable protrusion for tt fonts
}{}
\makeatletter
\@ifundefined{KOMAClassName}{% if non-KOMA class
  \IfFileExists{parskip.sty}{%
    \usepackage{parskip}
  }{% else
    \setlength{\parindent}{0pt}
    \setlength{\parskip}{6pt plus 2pt minus 1pt}}
}{% if KOMA class
  \KOMAoptions{parskip=half}}
\makeatother
\usepackage{xcolor}
\usepackage[margin=1in]{geometry}
\usepackage{color}
\usepackage{fancyvrb}
\newcommand{\VerbBar}{|}
\newcommand{\VERB}{\Verb[commandchars=\\\{\}]}
\DefineVerbatimEnvironment{Highlighting}{Verbatim}{commandchars=\\\{\}}
% Add ',fontsize=\small' for more characters per line
\usepackage{framed}
\definecolor{shadecolor}{RGB}{248,248,248}
\newenvironment{Shaded}{\begin{snugshade}}{\end{snugshade}}
\newcommand{\AlertTok}[1]{\textcolor[rgb]{0.94,0.16,0.16}{#1}}
\newcommand{\AnnotationTok}[1]{\textcolor[rgb]{0.56,0.35,0.01}{\textbf{\textit{#1}}}}
\newcommand{\AttributeTok}[1]{\textcolor[rgb]{0.13,0.29,0.53}{#1}}
\newcommand{\BaseNTok}[1]{\textcolor[rgb]{0.00,0.00,0.81}{#1}}
\newcommand{\BuiltInTok}[1]{#1}
\newcommand{\CharTok}[1]{\textcolor[rgb]{0.31,0.60,0.02}{#1}}
\newcommand{\CommentTok}[1]{\textcolor[rgb]{0.56,0.35,0.01}{\textit{#1}}}
\newcommand{\CommentVarTok}[1]{\textcolor[rgb]{0.56,0.35,0.01}{\textbf{\textit{#1}}}}
\newcommand{\ConstantTok}[1]{\textcolor[rgb]{0.56,0.35,0.01}{#1}}
\newcommand{\ControlFlowTok}[1]{\textcolor[rgb]{0.13,0.29,0.53}{\textbf{#1}}}
\newcommand{\DataTypeTok}[1]{\textcolor[rgb]{0.13,0.29,0.53}{#1}}
\newcommand{\DecValTok}[1]{\textcolor[rgb]{0.00,0.00,0.81}{#1}}
\newcommand{\DocumentationTok}[1]{\textcolor[rgb]{0.56,0.35,0.01}{\textbf{\textit{#1}}}}
\newcommand{\ErrorTok}[1]{\textcolor[rgb]{0.64,0.00,0.00}{\textbf{#1}}}
\newcommand{\ExtensionTok}[1]{#1}
\newcommand{\FloatTok}[1]{\textcolor[rgb]{0.00,0.00,0.81}{#1}}
\newcommand{\FunctionTok}[1]{\textcolor[rgb]{0.13,0.29,0.53}{\textbf{#1}}}
\newcommand{\ImportTok}[1]{#1}
\newcommand{\InformationTok}[1]{\textcolor[rgb]{0.56,0.35,0.01}{\textbf{\textit{#1}}}}
\newcommand{\KeywordTok}[1]{\textcolor[rgb]{0.13,0.29,0.53}{\textbf{#1}}}
\newcommand{\NormalTok}[1]{#1}
\newcommand{\OperatorTok}[1]{\textcolor[rgb]{0.81,0.36,0.00}{\textbf{#1}}}
\newcommand{\OtherTok}[1]{\textcolor[rgb]{0.56,0.35,0.01}{#1}}
\newcommand{\PreprocessorTok}[1]{\textcolor[rgb]{0.56,0.35,0.01}{\textit{#1}}}
\newcommand{\RegionMarkerTok}[1]{#1}
\newcommand{\SpecialCharTok}[1]{\textcolor[rgb]{0.81,0.36,0.00}{\textbf{#1}}}
\newcommand{\SpecialStringTok}[1]{\textcolor[rgb]{0.31,0.60,0.02}{#1}}
\newcommand{\StringTok}[1]{\textcolor[rgb]{0.31,0.60,0.02}{#1}}
\newcommand{\VariableTok}[1]{\textcolor[rgb]{0.00,0.00,0.00}{#1}}
\newcommand{\VerbatimStringTok}[1]{\textcolor[rgb]{0.31,0.60,0.02}{#1}}
\newcommand{\WarningTok}[1]{\textcolor[rgb]{0.56,0.35,0.01}{\textbf{\textit{#1}}}}
\usepackage{graphicx}
\makeatletter
\def\maxwidth{\ifdim\Gin@nat@width>\linewidth\linewidth\else\Gin@nat@width\fi}
\def\maxheight{\ifdim\Gin@nat@height>\textheight\textheight\else\Gin@nat@height\fi}
\makeatother
% Scale images if necessary, so that they will not overflow the page
% margins by default, and it is still possible to overwrite the defaults
% using explicit options in \includegraphics[width, height, ...]{}
\setkeys{Gin}{width=\maxwidth,height=\maxheight,keepaspectratio}
% Set default figure placement to htbp
\makeatletter
\def\fps@figure{htbp}
\makeatother
\setlength{\emergencystretch}{3em} % prevent overfull lines
\providecommand{\tightlist}{%
  \setlength{\itemsep}{0pt}\setlength{\parskip}{0pt}}
\setcounter{secnumdepth}{-\maxdimen} % remove section numbering
\ifLuaTeX
  \usepackage{selnolig}  % disable illegal ligatures
\fi
\IfFileExists{bookmark.sty}{\usepackage{bookmark}}{\usepackage{hyperref}}
\IfFileExists{xurl.sty}{\usepackage{xurl}}{} % add URL line breaks if available
\urlstyle{same}
\hypersetup{
  pdftitle={TP 2 echantillon},
  pdfauthor={Diego CASAS BARCENAS - Leo Jean UNITE},
  hidelinks,
  pdfcreator={LaTeX via pandoc}}

\title{TP 2 echantillon}
\author{Diego CASAS BARCENAS - Leo Jean UNITE}
\date{2024-11-19}

\begin{document}
\maketitle

\emph{Instructions}: Un compte-rendu du TP rédigé sous la forme d'un
fichier pdf est à rendre par binôme d'étudiants. Vous devrez déposer
votre compte rendu sous moodle. Chaque binôme remettra un document pdf
ayant pour nom \texttt{$nomsetudiants\_tptest.pdf$} avant le 17
décembre.

\hypertarget{exercice-1}{%
\subsection{Exercice 1}\label{exercice-1}}

L'objectif de cet exercice est d'utiliser des tests statistiques pour
comparer différents aliments en fonction d'indicateurs relatifs aux
impacts environnementaux de leur production agricole.

Nous allons travailler sous R avec des données extraites de la base de
données Agribalyse®. Agribalyse® est une base de données publique
française qui fournit des informations sur les impacts environnementaux
des produits agricoles et alimentaires. Pour en savoir plus, vous pouvez
télécharger la documentation à l'adresse suivante :
\url{https://doc.agribalyse.fr/documentation/acces-donnees}.

Le fichier Agri\_conv\_TP.xlsx sur lequel nous allons travailler dans ce
TP a été réalisé à partir des données extraites du tableur pour les
produits agricoles bruts conventionnels (à la sortie de la ferme) de la
base de données Agribalyse 3.1. Vous trouverez plus d'informations sur
la modification des noms des variables et la préparation des données
dans l'onglet ``Info'' du fichier Agri\_conv\_TP.xlsx. Ce fichier a été
élaboré par Julie Charles lors de son stage de BUT SD, en deuxième
année.

Nous disposons ici de valeurs de consommation en CO2 (ou équivalent) par
kg pour 258 aliments. De plus, pour chacun de ces aliments, nous avons
les informations suivantes :

\begin{itemize}
\tightlist
\item
  Catégorie : S'agit-il d'un produit d'origine animale ou végétale ?
\item
  Groupe : Quel est le sous-groupe du produit (par exemple, pour les
  produits animaux : bœuf, mouton, etc.) ?
\end{itemize}

\begin{enumerate}
\def\labelenumi{\arabic{enumi}.}
\setcounter{enumi}{-1}
\tightlist
\item
  Charger les données \emph{Agri\_conv\_TP.csv}.
\end{enumerate}

\begin{Shaded}
\begin{Highlighting}[]
\NormalTok{packages }\OtherTok{\textless{}{-}} \FunctionTok{c}\NormalTok{(}\StringTok{"dplyr"}\NormalTok{, }\StringTok{"readxl"}\NormalTok{, }\StringTok{"ggplot2"}\NormalTok{)}
                
\NormalTok{install\_if\_needed }\OtherTok{\textless{}{-}} \ControlFlowTok{function}\NormalTok{(pkg) \{                }
  \ControlFlowTok{if}\NormalTok{ (}\SpecialCharTok{!}\FunctionTok{require}\NormalTok{(pkg, }\AttributeTok{character.only =} \ConstantTok{TRUE}\NormalTok{)) \{}
    \FunctionTok{install.packages}\NormalTok{(pkg, }\AttributeTok{dependencies =} \ConstantTok{TRUE}\NormalTok{)}
\NormalTok{  \}}
  \FunctionTok{library}\NormalTok{(pkg, }\AttributeTok{character.only =} \ConstantTok{TRUE}\NormalTok{)}
\NormalTok{\}}

\FunctionTok{lapply}\NormalTok{(packages, install\_if\_needed)}
\end{Highlighting}
\end{Shaded}

\begin{verbatim}
## Le chargement a nécessité le package : dplyr
\end{verbatim}

\begin{verbatim}
## Warning: le package 'dplyr' a été compilé avec la version R 4.4.2
\end{verbatim}

\begin{verbatim}
## 
## Attachement du package : 'dplyr'
\end{verbatim}

\begin{verbatim}
## Les objets suivants sont masqués depuis 'package:stats':
## 
##     filter, lag
\end{verbatim}

\begin{verbatim}
## Les objets suivants sont masqués depuis 'package:base':
## 
##     intersect, setdiff, setequal, union
\end{verbatim}

\begin{verbatim}
## Le chargement a nécessité le package : readxl
\end{verbatim}

\begin{verbatim}
## Warning: le package 'readxl' a été compilé avec la version R 4.4.1
\end{verbatim}

\begin{verbatim}
## Le chargement a nécessité le package : ggplot2
\end{verbatim}

\begin{verbatim}
## Warning: le package 'ggplot2' a été compilé avec la version R 4.4.1
\end{verbatim}

\begin{verbatim}
## [[1]]
## [1] "dplyr"     "stats"     "graphics"  "grDevices" "utils"     "datasets" 
## [7] "methods"   "base"     
## 
## [[2]]
## [1] "readxl"    "dplyr"     "stats"     "graphics"  "grDevices" "utils"    
## [7] "datasets"  "methods"   "base"     
## 
## [[3]]
##  [1] "ggplot2"   "readxl"    "dplyr"     "stats"     "graphics"  "grDevices"
##  [7] "utils"     "datasets"  "methods"   "base"
\end{verbatim}

\begin{Shaded}
\begin{Highlighting}[]
\NormalTok{data }\OtherTok{\textless{}{-}} \FunctionTok{read\_excel}\NormalTok{(}\StringTok{"Agri\_conv\_TP.xlsx"}\NormalTok{)}
\FunctionTok{print}\NormalTok{(data)}
\end{Highlighting}
\end{Shaded}

\begin{verbatim}
## # A tibble: 258 x 17
##    Nom_produit          Co2_eq Appauvr_Ozone Rayon Ozone_chimie Particules_fines
##    <chr>                 <dbl>         <dbl> <dbl>        <dbl>            <dbl>
##  1 Homard, 1 kg de pro~  14.2    0.00000314  1.53        0.190       0.00000219 
##  2 Thon albacore, AEN,~   4.19   0.000000950 0.259       0.0999      0.00000109 
##  3 Saithe, Mer du Nord~   4.24   0.00000113  0.246       0.0928      0.00000102 
##  4 Gadidae, Mer Celtiq~   3.27   0.000000724 0.202       0.0760      0.000000833
##  5 Tourteau, 1 kg de p~   4.56   0.000000928 0.526       0.0597      0.000000688
##  6 Tourteau, 1 kg de p~   4.16   0.000000836 0.362       0.0537      0.000000618
##  7 Saithe, Mer du Nord~   2.07   0.000000488 0.126       0.0479      0.000000525
##  8 Sole commune, BBisc~   2.15   0.000000433 0.125       0.0452      0.000000498
##  9 Thon Albacore, la C~   1.89   0.000000413 0.120       0.0437      0.000000478
## 10 Thon Listao, la CEA~   1.88   0.000000411 0.119       0.0434      0.000000476
## # i 248 more rows
## # i 11 more variables: Acidification <dbl>, Eutrophisation_eaux_douces <dbl>,
## #   Eutrophisation_marine <dbl>, Eutrophisation_terrestre <dbl>,
## #   Utilisation_du_sol <dbl>, Epuisement_eau <dbl>, Epuisement_energie <dbl>,
## #   Epuisement_minéraux <dbl>, Categorie <chr>, Groupe <chr>, Score_EF <dbl>
\end{verbatim}

\begin{enumerate}
\def\labelenumi{\arabic{enumi}.}
\tightlist
\item
  On s'interesse au Co2\_eq. Faire une étude réprésentation univariée de
  cette variable.
\end{enumerate}

\begin{Shaded}
\begin{Highlighting}[]
\NormalTok{stat }\OtherTok{\textless{}{-}}\NormalTok{ data }\SpecialCharTok{\%\textgreater{}\%} 
  \FunctionTok{summarise}\NormalTok{(}
    \StringTok{"NB Quantité"}\OtherTok{=}\FunctionTok{n}\NormalTok{(),}
    \FunctionTok{min}\NormalTok{(Co2\_eq),}
    \FunctionTok{max}\NormalTok{(Co2\_eq),}
    \FunctionTok{mean}\NormalTok{(Co2\_eq),}
    \FunctionTok{median}\NormalTok{(Co2\_eq),}
    \AttributeTok{EcartType =} \FunctionTok{sd}\NormalTok{(Co2\_eq)}
\NormalTok{  )}
\FunctionTok{print}\NormalTok{(stat)}
\end{Highlighting}
\end{Shaded}

\begin{verbatim}
## # A tibble: 1 x 6
##   `NB Quantité` `min(Co2_eq)` `max(Co2_eq)` `mean(Co2_eq)` `median(Co2_eq)`
##           <int>         <dbl>         <dbl>          <dbl>            <dbl>
## 1           258        0.0386          18.7           2.36            0.628
## # i 1 more variable: EcartType <dbl>
\end{verbatim}

\begin{Shaded}
\begin{Highlighting}[]
\FunctionTok{cat}\NormalTok{(}\StringTok{"test"}\NormalTok{, }\FunctionTok{min}\NormalTok{(stat)) }\CommentTok{\# Rediger un texte dans le code.}
\end{Highlighting}
\end{Shaded}

\begin{verbatim}
## test 0.03855912
\end{verbatim}

\begin{enumerate}
\def\labelenumi{\arabic{enumi}.}
\setcounter{enumi}{1}
\tightlist
\item
  On s'intéresse ici au lien entre la catégorie de produit (animale ou
  végétale) et le Co2.
\end{enumerate}

\begin{Shaded}
\begin{Highlighting}[]
\NormalTok{Animal }\OtherTok{\textless{}{-}}\NormalTok{ data }\SpecialCharTok{\%\textgreater{}\%} 
  \FunctionTok{group\_by}\NormalTok{(Categorie) }\SpecialCharTok{\%\textgreater{}\%}
  \FunctionTok{filter}\NormalTok{(Categorie }\SpecialCharTok{==} \StringTok{"Animal production"}\NormalTok{) }\SpecialCharTok{\%\textgreater{}\%} 
  \FunctionTok{summarise}\NormalTok{(}
    \StringTok{"NB Quantité"}\OtherTok{=}\FunctionTok{n}\NormalTok{(),}
    \FunctionTok{min}\NormalTok{(Co2\_eq),}
    \FunctionTok{max}\NormalTok{(Co2\_eq),}
    \FunctionTok{mean}\NormalTok{(Co2\_eq),}
    \FunctionTok{median}\NormalTok{(Co2\_eq),}
    \AttributeTok{EcartType =} \FunctionTok{sd}\NormalTok{(Co2\_eq)}
\NormalTok{  )}
\FunctionTok{print}\NormalTok{(Animal)}
\end{Highlighting}
\end{Shaded}

\begin{verbatim}
## # A tibble: 1 x 7
##   Categorie         `NB Quantité` `min(Co2_eq)` `max(Co2_eq)` `mean(Co2_eq)`
##   <chr>                     <int>         <dbl>         <dbl>          <dbl>
## 1 Animal production            88         0.137          18.7           5.61
## # i 2 more variables: `median(Co2_eq)` <dbl>, EcartType <dbl>
\end{verbatim}

\begin{Shaded}
\begin{Highlighting}[]
\NormalTok{Plantes }\OtherTok{\textless{}{-}}\NormalTok{ data }\SpecialCharTok{\%\textgreater{}\%} 
  \FunctionTok{group\_by}\NormalTok{(Categorie) }\SpecialCharTok{\%\textgreater{}\%}
  \FunctionTok{filter}\NormalTok{(Categorie }\SpecialCharTok{==} \StringTok{"Plant production"}\NormalTok{) }\SpecialCharTok{\%\textgreater{}\%} 
  \FunctionTok{summarise}\NormalTok{(}
    \StringTok{"NB Quantité"}\OtherTok{=}\FunctionTok{n}\NormalTok{(),}
    \FunctionTok{min}\NormalTok{(Co2\_eq),}
    \FunctionTok{max}\NormalTok{(Co2\_eq),}
    \FunctionTok{mean}\NormalTok{(Co2\_eq),}
    \FunctionTok{median}\NormalTok{(Co2\_eq),}
    \AttributeTok{EcartType =} \FunctionTok{sd}\NormalTok{(Co2\_eq)}
\NormalTok{  )}
\FunctionTok{print}\NormalTok{(Plantes)}
\end{Highlighting}
\end{Shaded}

\begin{verbatim}
## # A tibble: 1 x 7
##   Categorie        `NB Quantité` `min(Co2_eq)` `max(Co2_eq)` `mean(Co2_eq)`
##   <chr>                    <int>         <dbl>         <dbl>          <dbl>
## 1 Plant production           170        0.0386          7.23          0.681
## # i 2 more variables: `median(Co2_eq)` <dbl>, EcartType <dbl>
\end{verbatim}

\begin{enumerate}
\def\labelenumi{\alph{enumi}.}
\tightlist
\item
  Etudier de manière descriptive le lien entre la variable catégorie et
  le Co2 (avec un graphique pertinent)
\end{enumerate}

\begin{Shaded}
\begin{Highlighting}[]
\FunctionTok{ggplot}\NormalTok{(data, }\FunctionTok{aes}\NormalTok{(}\AttributeTok{x =}\NormalTok{ Categorie, }\AttributeTok{y =}\NormalTok{ Co2\_eq))}\SpecialCharTok{+}
  \FunctionTok{geom\_boxplot}\NormalTok{()}
\end{Highlighting}
\end{Shaded}

\includegraphics{TP2_2ech_files/figure-latex/unnamed-chunk-1-1.pdf}

\begin{Shaded}
\begin{Highlighting}[]
\FunctionTok{cat}\NormalTok{(}\StringTok{"En vue des résultats obtenus, l\textquotesingle{}impacte de CO2 est plus importante chez les animaux, en particulier à cause des flatulence animale."}\NormalTok{)}
\end{Highlighting}
\end{Shaded}

\begin{verbatim}
## En vue des résultats obtenus, l'impacte de CO2 est plus importante chez les animaux, en particulier à cause des flatulence animale.
\end{verbatim}

\begin{enumerate}
\def\labelenumi{\alph{enumi}.}
\setcounter{enumi}{1}
\item
  On veut maintenant tester si la moyenne de Co2 émis est la même pour
  les aliments animaux et les aliments végétaux. Préciser dans quelle
  situation vous vous trouvez (taille de l'échantillon, normalité
  égalité des variances,\ldots). Puis faites le test et concluez.
\item
  Si on voulait montrer que les produits animaux ont une espérance plus
  élevée que les produits végétaux que faudrait il changer (refaites la
  question précédente dans ce sens). Faites une conclusion.
\end{enumerate}

\begin{enumerate}
\def\labelenumi{\arabic{enumi}.}
\setcounter{enumi}{2}
\tightlist
\item
  On veut mettre en évidence au risque 5\% que la proportion d'aliment
  végétal qui nécessitent plus d'un kilo d'équivalent CO2 pour un kilo
  de production est inférieur chez les végétaux que chez les animaux.
  Faites le test et donner la conclusion adéquate.
\end{enumerate}

\hypertarget{exercice-2-lamour-ouf-fait-le-buzz.}{%
\subsection{Exercice 2 : l'amour ouf fait le
buzz.}\label{exercice-2-lamour-ouf-fait-le-buzz.}}

L'amour Ouf a été décrit comme très populaire auprès des 15-25 ans. Afin
d'étudier ce phénomène, l'Institut Vertigo (un institut d'études
spécialisé dans la recherche marketing pour le marché des loisirs) a
mené une enquête sur l'âge des spectateurs à la sortie de certains films
en 2024. Les résultats obtenus sont les suivants :

\begin{Shaded}
\begin{Highlighting}[]
\FunctionTok{data.frame}\NormalTok{(}\AttributeTok{Film =} \FunctionTok{c}\NormalTok{( }\StringTok{"Amour Ouf"}\NormalTok{, }\StringTok{"Dead\_pool"}\NormalTok{,}\StringTok{"Dune2"}\NormalTok{,}\StringTok{"Monte\_Cristo"}\NormalTok{, }\StringTok{"Emilia\_Perez"}\NormalTok{), }\AttributeTok{Age\_15\_25=} \FunctionTok{c}\NormalTok{(}\FloatTok{0.29}\NormalTok{,}\FloatTok{0.27}\NormalTok{,}\FloatTok{0.25}\NormalTok{,}\FloatTok{0.2}\NormalTok{,}\FloatTok{0.11}\NormalTok{)}\SpecialCharTok{*}\DecValTok{2000}\NormalTok{ , }\AttributeTok{Age\_Autres =}\NormalTok{ (}\DecValTok{1}\SpecialCharTok{{-}}\FunctionTok{c}\NormalTok{(}\FloatTok{0.29}\NormalTok{,}\FloatTok{0.27}\NormalTok{,}\FloatTok{0.25}\NormalTok{,}\FloatTok{0.2}\NormalTok{,}\FloatTok{0.11}\NormalTok{))}\SpecialCharTok{*}\DecValTok{2000}\NormalTok{ )}
\end{Highlighting}
\end{Shaded}

\begin{verbatim}
##           Film Age_15_25 Age_Autres
## 1    Amour Ouf       580       1420
## 2    Dead_pool       540       1460
## 3        Dune2       500       1500
## 4 Monte_Cristo       400       1600
## 5 Emilia_Perez       220       1780
\end{verbatim}

\begin{enumerate}
\def\labelenumi{\arabic{enumi}.}
\item
  Au risque de 5\%, pouvez-vous dire que la proportion de 15-25 ans
  allant voir un film varie en fonction du film ?
\item
  Au risque de 5\%, pouvez-vous dire que la proportion de spectateurs
  âgés de 15-25 ans ayant vu ``Amour Ouf'' est significativement
  supérieure à celle des autres films étudiés (tous films confondus) ?
\item
  En 2024, 20\% des spectateurs de cinéma ont entre 15 et 25 ans. a. Si
  20\% des spectateurs de ``Amour Ouf'' avaient eu entre 15 et 25 ans,
  combien aurait-on observé théoriquement de personnes de cette tranche
  d'âge parmi les spectateurs interrogés ayant vu ce film ?
\end{enumerate}

\begin{enumerate}
\def\labelenumi{\alph{enumi}.}
\setcounter{enumi}{1}
\tightlist
\item
  Proposez un test permettant de comparer les effectifs observés dans la
  salle de ``Amour Ouf'' à l'effectif théorique si 20\% des spectateurs
  avaient eu entre 15 et 25 ans (test du chi carré d'adéquation à une
  loi).
\end{enumerate}

\end{document}
